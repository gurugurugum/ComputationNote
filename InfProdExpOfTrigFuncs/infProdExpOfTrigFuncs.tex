\documentclass[10pt,onecolumn,dvipdfmx]{jsarticle}
\usepackage{amsmath,amssymb,float,braket,bm,moreverb,mediabb,hyperref}
\usepackage[dvips]{pict2e}

\DeclareMathOperator{\sinc}{sinc}
\DeclareMathOperator{\diag}{diag}

\DeclareSymbolFont{lettersA}{U}{txmia}{m}{it}
\DeclareMathSymbol{\F}{\mathord}{lettersA}{'206}
\DeclareMathSymbol{\N}{\mathord}{lettersA}{'216} 
\DeclareMathSymbol{\R}{\mathord}{lettersA}{'222} 
\DeclareMathSymbol{\Z}{\mathord}{lettersA}{'232} 
\DeclareMathSymbol{\Q}{\mathord}{lettersA}{'221} 
\DeclareMathSymbol{\C}{\mathord}{lettersA}{'203} 

\newcommand{\divergence}{\mathrm{div}\,} 
\newcommand{\grad}{\mathrm{grad}\,} 
\newcommand{\rot}{\mathrm{rot}\,} 

\def\sla#1{\rlap{\kern .15em /}#1}

\newcommand{\diff}[3]{\frac{\mathrm{d} ^{#1} #2}{\mathrm{d} #3^{#1}} }
\newcommand{\pdiff}[3]{\frac{\partial ^{#1} #2}{\partial #3^{#1}} }

\renewcommand{\thesubsection}{\arabic{subsection})}

\renewcommand{\theequation}{\arabic{section}.\arabic{equation}}
\makeatletter
\@addtoreset{equation}{subsection}
\makeatother

\makeatletter
\renewcommand{\paragraph}{\@startsection{paragraph}{4}{\z@}%
  {1.5\Cvs \@plus.5\Cdp \@minus.2\Cdp}%
  {.5\Cvs \@plus.3\Cdp}%
  {\reset@font\normalsize\bfseries}}
\makeatother

\begin{document}

\title{三角関数の無限乗積展開}
\author{岡 和磨}
\date{2016/12/31}
\maketitle

\section{公式}
\begin{align}
\sin{\left( \pi z\right) } &= \pi z\prod_{n=1} ^{\infty } \left( 1 - \frac{z^2 }{n^2 } \right) \\
\cos{\left( \pi z\right) } &= \prod_{n=1} ^{\infty } \left( 1 - \frac{z^2 }{\left( n - 1/2\right) ^2 } \right) \\
\sinh{\left( \pi z\right) } &= \pi z\prod_{n=1} ^{\infty } \left( 1 + \frac{z^2 }{n^2 } \right) \\
\cosh{\left( \pi z\right) } &= \prod_{n=1} ^{\infty } \left( 1 + \frac{z^2 }{\left( n - 1/2\right) ^2 } \right) 
\end{align}

\section{証明}
\begin{align}
f\left( z\right) = \frac{\pi z\prod_{n=1} ^{\infty } \left( 1 - \frac{z^2 }{n^2 } \right) }{\sin{\left( \pi z\right) } } 
\end{align}
とおく。
\begin{align}
\frac{d}{dz} \log{\left( f\left( z\right) \right) } &= \sum_{n=-\infty } ^{\infty } \frac{1}{z + n} - \pi \cot{\left( \pi z\right) } \notag \\
&= 0\notag \\
\therefore f\left( z\right) &= f\left( 0\right) =1\notag \\
\therefore \sin{\left( \pi z\right) } &= \pi z\prod_{n=1} ^{\infty } \left( 1 - \frac{z^2 }{n^2 } \right) 
\end{align}
同様に、
\begin{align}
g\left( z\right) = \frac{\prod_{n=1} ^{\infty } \left( 1 - \frac{z^2 }{\left( n - 1/2\right) ^2 } \right) }{\cos{\left( \pi z\right) } } 
\end{align}
とすると、
\begin{align}
\frac{d}{dz} \log{\left( g\left( z\right) \right) } &= \sum_{n=-\infty } ^{\infty } \frac{1}{z + n + 1 / 2} + \pi \tan{\left( \pi z\right) } \notag \\
&= 0\notag \\
\therefore g\left( z\right) &= g\left( 0\right) =1\notag \\
\therefore \cos{\left( \pi z\right) } &= \prod_{n=1} ^{\infty } \left( 1 - \frac{z^2 }{\left( n - 1/2\right) ^2 } \right) 
\end{align}

\begin{thebibliography}{9}
\bibitem{wikiInfProdTrigFunc} Wikipedia 三角関数の無限乗積展開 (\href{https://ja.wikipedia.org/wiki/%E4%B8%89%E8%A7%92%E9%96%A2%E6%95%B0%E3%81%AE%E7%84%A1%E9%99%90%E4%B9%97%E7%A9%8D%E5%B1%95%E9%96%8B}{https://ja.wikipedia.org/wiki/三角関数の無限乗積展開}) アクセス日時:2016年12月31日13時
\end{thebibliography}

\end{document}