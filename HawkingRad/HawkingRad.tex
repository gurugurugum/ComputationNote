\documentclass[10pt,onecolumn,dvipdfmx]{jsarticle}
\usepackage{amsmath,amssymb,float,braket,bm,moreverb,mediabb,hyperref}
\usepackage[dvips]{pict2e}

\DeclareMathOperator{\sinc}{sinc}
\DeclareMathOperator{\diag}{diag}

\DeclareSymbolFont{lettersA}{U}{txmia}{m}{it}
\DeclareMathSymbol{\F}{\mathord}{lettersA}{'206}
\DeclareMathSymbol{\N}{\mathord}{lettersA}{'216} 
\DeclareMathSymbol{\R}{\mathord}{lettersA}{'222} 
\DeclareMathSymbol{\Z}{\mathord}{lettersA}{'232} 
\DeclareMathSymbol{\Q}{\mathord}{lettersA}{'221} 
\DeclareMathSymbol{\C}{\mathord}{lettersA}{'203} 

\newcommand{\divergence}{\mathrm{div}\,} 
\newcommand{\grad}{\mathrm{grad}\,} 
\newcommand{\rot}{\mathrm{rot}\,} 

\def\sla#1{\rlap{\kern .15em /}#1}

\newcommand{\diff}[3]{\frac{\mathrm{d} ^{#1} #2}{\mathrm{d} #3^{#1}} }
\newcommand{\pdiff}[3]{\frac{\partial ^{#1} #2}{\partial #3^{#1}} }

\renewcommand{\thesubsection}{\arabic{subsection})}

\renewcommand{\theequation}{\arabic{section}.\arabic{equation}}
\makeatletter
\@addtoreset{equation}{subsection}
\makeatother

\makeatletter
\renewcommand{\paragraph}{\@startsection{paragraph}{4}{\z@}%
  {1.5\Cvs \@plus.5\Cdp \@minus.2\Cdp}%
  {.5\Cvs \@plus.3\Cdp}%
  {\reset@font\normalsize\bfseries}}
\makeatother

\begin{document}

\title{}
\author{Kazuma Oka}
\date{}
\maketitle

\section{Schwarzschild black hole}
Firstly, we consider a metric
\begin{align}
\mathrm{d} s^2 &=-\left( 1-\frac{2GM}{r} \right) \mathrm{d} u\mathrm{d} v+r^2 \mathrm{d} \theta ^2 +r^2 \sin^2 {\theta } \mathrm{d} \phi ^2 \\
u=&t-r^*, v=t+r^*, r^*=r+2GM\ln{\left( \frac{r}{2GM} -1\right) }
\end{align}


\begin{thebibliography}{9}
\bibitem{gravAndEnt} 福間 将文, 酒谷雄峰, 『重力とエントロピー 〜 重力の熱力学的性質を理解するために 〜』(SGCライブラリ90), サイエンス社 (2014).
\end{thebibliography}

\end{document}